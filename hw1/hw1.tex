%%%%%%%%%%%%%%%%%%%%%%%%%%%%%%%%%%%%%%%%%
% Structured General Purpose Assignment
% LaTeX Template
%
% This template has been downloaded from:
% http://www.latextemplates.com
%
% Original author:
% Ted Pavlic (http://www.tedpavlic.com)
%
% Note:
% The \lipsum[#] commands throughout this template generate dummy text
% to fill the template out. These commands should all be removed when 
% writing assignment content.
%
%%%%%%%%%%%%%%%%%%%%%%%%%%%%%%%%%%%%%%%%%

%----------------------------------------------------------------------------------------
%	PACKAGES AND OTHER DOCUMENT CONFIGURATIONS
%----------------------------------------------------------------------------------------

\documentclass{article}

\usepackage{fancyhdr} % Required for custom headers
\usepackage{lastpage} % Required to determine the last page for the footer
\usepackage{extramarks} % Required for headers and footers
\usepackage{graphicx} % Required to insert images
\usepackage{lipsum} % Used for inserting dummy 'Lorem ipsum' text into the template
\usepackage{listings}
\usepackage{color}
\usepackage{amsmath}

\definecolor{dkgreen}{rgb}{0,0.6,0}
\definecolor{gray}{rgb}{0.5,0.5,0.5}
\definecolor{mauve}{rgb}{0.58,0,0.82}

\lstset{frame=tb,
  language=R,
  aboveskip=3mm,
  belowskip=3mm,
  showstringspaces=false,
  columns=flexible,
  basicstyle={\small\ttfamily},
  numbers=none,
  numberstyle=\tiny\color{gray},
  keywordstyle=\color{blue},
  commentstyle=\color{dkgreen},
  stringstyle=\color{mauve},
  breaklines=true,
  breakatwhitespace=true
  tabsize=3
}

% Margins
\topmargin=-0.45in
\evensidemargin=0in
\oddsidemargin=0in
\textwidth=6.5in
\textheight=9.0in
\headsep=0.25in 

\linespread{1.1} % Line spacing

% Set up the header and footer
\pagestyle{fancy}
\lhead{\hmwkAuthorName} % Top left header
\chead{\hmwkClass\ (\hmwkClassInstructor\ \hmwkClassTime): \hmwkTitle} % Top center header
\rhead{\firstxmark} % Top right header
\lfoot{\lastxmark} % Bottom left footer
\cfoot{} % Bottom center footer
\rfoot{Page\ \thepage\ of\ \pageref{LastPage}} % Bottom right footer
\renewcommand\headrulewidth{0.4pt} % Size of the header rule
\renewcommand\footrulewidth{0.4pt} % Size of the footer rule

\setlength\parindent{0pt} % Removes all indentation from paragraphs

%----------------------------------------------------------------------------------------
%	DOCUMENT STRUCTURE COMMANDS
%	Skip this unless you know what you're doing
%----------------------------------------------------------------------------------------

% Header and footer for when a page split occurs within a problem environment
\newcommand{\enterProblemHeader}[1]{
\nobreak\extramarks{#1}{#1 continued on next page\ldots}\nobreak
\nobreak\extramarks{#1 (continued)}{#1 continued on next page\ldots}\nobreak
}

% Header and footer for when a page split occurs between problem environments
\newcommand{\exitProblemHeader}[1]{
\nobreak\extramarks{#1 (continued)}{#1 continued on next page\ldots}\nobreak
\nobreak\extramarks{#1}{}\nobreak
}

\setcounter{secnumdepth}{0} % Removes default section numbers
\newcounter{homeworkProblemCounter} % Creates a counter to keep track of the number of problems

\newcommand{\homeworkProblemName}{}
\newenvironment{homeworkProblem}[1][Problem \arabic{homeworkProblemCounter}]{ % Makes a new environment called homeworkProblem which takes 1 argument (custom name) but the default is "Problem #"
\stepcounter{homeworkProblemCounter} % Increase counter for number of problems
\renewcommand{\homeworkProblemName}{#1} % Assign \homeworkProblemName the name of the problem
\section{\homeworkProblemName} % Make a section in the document with the custom problem count
\enterProblemHeader{\homeworkProblemName} % Header and footer within the environment
}{
\exitProblemHeader{\homeworkProblemName} % Header and footer after the environment
}

\newcommand{\problemAnswer}[1]{ % Defines the problem answer command with the content as the only argument
\noindent\framebox[\columnwidth][c]{\begin{minipage}{0.98\columnwidth}#1\end{minipage}} % Makes the box around the problem answer and puts the content inside
}

\newcommand{\homeworkSectionName}{}
\newenvironment{homeworkSection}[1]{ % New environment for sections within homework problems, takes 1 argument - the name of the section
\renewcommand{\homeworkSectionName}{#1} % Assign \homeworkSectionName to the name of the section from the environment argument
\subsection{\homeworkSectionName} % Make a subsection with the custom name of the subsection
\enterProblemHeader{\homeworkProblemName\ [\homeworkSectionName]} % Header and footer within the environment
}{
\enterProblemHeader{\homeworkProblemName} % Header and footer after the environment
}
   
%----------------------------------------------------------------------------------------
%	NAME AND CLASS SECTION
%----------------------------------------------------------------------------------------

\newcommand{\hmwkTitle}{Hw1} % Assignment title
\newcommand{\hmwkDueDate}{July 24,\ 2014} % Due date
\newcommand{\hmwkClass}{Probability and Elements of Real Analysis} % Course/class
\newcommand{\hmwkClassTime}{} % Class/lecture time
\newcommand{\hmwkClassInstructor}{Instructor: Elena} % Teacher/lecturer
\newcommand{\hmwkAuthorName}{Weiyi Chen} % Your name

%----------------------------------------------------------------------------------------
%	TITLE PAGE
%----------------------------------------------------------------------------------------

\title{
\vspace{2in}
\textmd{\textbf{\hmwkClass:\ \hmwkTitle}}\\
\normalsize\vspace{0.1in}\small{Due\ on\ \hmwkDueDate}\\
\vspace{0.1in}\large{\textit{\hmwkClassInstructor\ \hmwkClassTime}}
\vspace{3in}
}

\author{\textbf{\hmwkAuthorName}}
\date{} % Insert date here if you want it to appear below your name

%----------------------------------------------------------------------------------------

\begin{document}

\maketitle

%----------------------------------------------------------------------------------------
%	TABLE OF CONTENTS
%----------------------------------------------------------------------------------------

%\setcounter{tocdepth}{1} % Uncomment this line if you don't want subsections listed in the ToC

%\newpage
%\tableofcontents
\newpage

%----------------------------------------------------------------------------------------
%	PROBLEM 1
%----------------------------------------------------------------------------------------

\begin{homeworkProblem}
    According to the countable additivity of definition 1.1.2, let
    \begin{equation}
        \begin{split}
            A_1 &= \Omega \\
            A_n &= \phi \text{ for } n \in Z^+/\{1\}
        \end{split}
    \end{equation}
    then
    \begin{equation}
        P(\Omega) = P(\bigcup_{n=1}^{\infty} A_n) = \sum_{n=1}^{\infty} P(A_n) = P(\Omega) + \sum_{n=2}^{\infty} P(A_n)
    \end{equation}
    Therefore we have $\sum_{n=2}^{\infty} P(A_n) = 0$ which implies $P(\phi) = 0$ since probability is always non-negative. \\
    Using this proposition and countable additivity again, let
    \begin{equation}
        A_n = \phi \text{ for } n \in Z^+/\{1, 2, 3, ..., k\}
    \end{equation}
    We derive the finite additivity as
    \begin{equation}
        P(\bigcup_{n=1}^{k} A_n) = P(\bigcup_{n=1}^{\infty} A_n) = \sum_{n=1}^{\infty} P(A_n) = \sum_{n=1}^{k} P(A_n)
    \end{equation}
    \begin{homeworkSection}{(i)}
        We apply finite additivity to prove this problem,
        \begin{equation}
            P(B) = P((B - A) \cup A) = P(B - A) + P(A) \ge P(A).
        \end{equation}
    \end{homeworkSection}
    \begin{homeworkSection}{(ii)}
        Using the conclusion of part (i), since $A \subset A_n$, then
        \begin{equation}
            P(A) \le P(A_n)~\forall n \in Z^+
        \end{equation}
        Therefore
        \begin{equation}
            P(A) \le \lim_{n \to \infty}{P(A_n)} = 0 \Rightarrow P(A) = 0
        \end{equation}
    \end{homeworkSection}
\end{homeworkProblem}

%----------------------------------------------------------------------------------------
%   PROBLEM 2
%----------------------------------------------------------------------------------------

\begin{homeworkProblem}
    First we construct a uniformly distributed random variable taking values in [0,1] and defined on infinite coin-space $\Omega_{\infty}$, as the way Example 1.2.5 did. According to the assumption of this problem, the probability for head on each toss is $p = \frac{1}{2}$. For $n \in Z^+$,
    \begin{equation}
        Y_n(\omega) := 
        \begin{cases}
            1 &\text{ if } \omega_n = H \\
            0 &\text{ if } \omega_n = T
        \end{cases}      
    \end{equation}
    We set
    \begin{equation}
        X = \sum_{n=1}^{\infty} \frac{Y_n}{2^n}
    \end{equation}
    In terms of the distribution measure $\mu_X$ of X, we write it as
    \begin{equation}
        \mu_X[\frac{k}{2^n}, \frac{k+1}{2^n}] = \frac{1}{2^n}
    \end{equation}
    for all k and n are integers such that $0 \le k \le 2^n - 1$. It can be shown that
    \begin{equation}
        \mu_X[a,b] = b - a
    \end{equation}
    for $0 \le a \le b \le 1$. That is, the distribution of X is uniform on [0,1].
    \begin{homeworkSection}{(i)}
        We consider cdf of normal distribution,
        \begin{equation}
            N(x) = \int_{-\infty}^{x} \frac{1}{\sqrt{2\pi}} e^{-\frac{\xi^2}{2}} d\xi
        \end{equation}
        We construct random variable $Z$ as $Z = N^{-1}(X)$, then
        \begin{equation}
           P(Z \le a) = P(X \le N(a)) = N(a)
        \end{equation}
        for any real number a. That is, $Z$ is a standard normal random variable on the coin-toss space $(\Omega_{\infty}, \cal{F}_{\infty}, P)$.
    \end{homeworkSection}
    \begin{homeworkSection}{(ii)}
        We define $Z$ as
        \begin{equation}
            Z := N^{-1}(X)
        \end{equation}
        Then
        \begin{equation}
            \lim_{n \to \infty} Z_n(\omega) = \lim_{n \to \infty} N^{-1}(X_n(\omega)) = N^{-1}(\lim_{n \to \infty}X_n(\omega)) = N^{-1}(X(\omega)) = Z(\omega)
        \end{equation}
        for every $\omega \in \Omega_{\infty}$. Therefore $\{Z_n\}_{n=1}^{\infty}$ is the expected sequence.
    \end{homeworkSection}
\end{homeworkProblem}

%----------------------------------------------------------------------------------------
%   PROBLEM 3
%----------------------------------------------------------------------------------------

\begin{homeworkProblem}
    To derive the smallest algebra contains the union of the two given algebra $\cal{F_1} = {\phi, U, U^c, \Omega}$ and $\cal{F_2} = {\phi, V, V^c, \Omega}$, we need to generate all the atoms. An atom of $\cal{F}$ is a set $A \in \cal{F}$ such that the only subsets of A which are also in $\cal{F}$ are the empty set $\phi$ and A itself. \\
    Therefore in our problem, we have four atoms at most, which are
    \begin{equation}
        A_1 = U \cap V^c, A_2 = V \cap U^c, A_3 = U \cap V, A_4 = U^c \cap V^c 
    \end{equation}
    This can be easily observed from a Wayne Figure. In general the smallest algebra is just the power set of these four sets, that is
    \begin{equation}
        \begin{split}
            \cal{F} = \{&\phi, \\
            &A_1, A_2, A_3, A_4, \\
            &A_1 \cup A_2, A_1 \cup A_3, A_1 \cup A_4, A_2 \cup A_3, A_2 \cup A_4, A_3 \cup A_4, \\
            &A_1 \cup A_2 \cup A_3, A_1 \cup A_2 \cup A_4, A_1 \cup A_3 \cup A_4, A_2 \cup A_3 \cup A_4, \\
            &\Omega\}
        \end{split}
    \end{equation}
    But above is the most general case, which implies that $U \cap V \neq \phi$ and $U \cup V \neq \Omega$. There are 3 more cases,
    \begin{itemize}
        \item $U \cap V \neq \phi$ and $U \cup V = \Omega$ \\
        3 atoms: $A_1 = U \cap V^c, A_2 = V \cap U^c, A_3 = U \cap V$ \\
        $2^3 = 8$ elements: $\cal{F}$ $= \{\phi, A_1, A_2, A_3, A_1 \cup A_2, A_1 \cup A_3, A_2 \cup A_3, \Omega\}$
        \item $U \cap V = \phi$ and $U \cup V \neq \Omega$
        3 atoms: $A_1 = U, A_2 = V, A_3 = U^c \cap V^c$ \\
        $2^3 = 8$ elements: $\cal{F}$ $= \{\phi, A_1, A_2, A_3, A_1 \cup A_2, A_1 \cup A_3, A_2 \cup A_3, \Omega\}$
        \item $U \cap V = \phi$ and $U \cup V = \Omega$ \\
        2 atoms: $A_1 = U, A_2 = V = U^c$ \\
        $2^2 = 4$ elements: $\cal{F}$ $= \{\phi, A_1, A_2, \Omega\}$
    \end{itemize}
\end{homeworkProblem}

\end{document}